%%%%%%%%%%%%%%%%%%%%%%%%%%%%%%%%%%%%%%%%%%%%%%%%%%%%%%%%%%%%%%%%%%%%%%%%%%%
%
% Template for a LaTex article in English.
%
%%%%%%%%%%%%%%%%%%%%%%%%%%%%%%%%%%%%%%%%%%%%%%%%%%%%%%%%%%%%%%%%%%%%%%%%%%%

\documentclass[toc=sectionentrywithdots]{scrartcl}

% AMS packages:
\usepackage{amsmath, amsthm, amsfonts} %Mathe-Pakete
\usepackage[colorlinks=false, pdfborder={0 0 0}]{hyperref} %Klickbare Referenzierung
\usepackage[hang,flushmargin]{footmisc} %Manipulation der Footnotes
\usepackage[square,sort,comma,numbers]{natbib} %Literaturverzeichnis mit Vancouver
\usepackage[ngerman]{babel}
\usepackage[utf8]{inputenc} %Zeichenerkennung
\usepackage{geometry} %Seiten Layout Paket
\geometry{a4paper,left=40mm,right=40mm, top=1cm, bottom=2cm} %Manipulation Layout 

% Change font of title and section
\setkomafont{disposition}{\normalfont}
%\addtokomafont{disposition}{\rmfamily}
\usepackage{setspace}%zeilenabstand
\onehalfspacing %1,5-facher zeilenabstand
%\usepackage{lmodern}
\usepackage{microtype}%wichtig für Blockabsatz
\usepackage{alnumsec}
\usepackage{anysize}
\usepackage{parskip}%Einzug&Absatz
\setlength{\parindent}{1em}%mit Einzug
\usepackage{graphicx} %Grafiken
\usepackage{float} %Float-Umgebung
\usepackage{longtable} %Tabellen über zwei Seiten
\usepackage{caption} %Float ohne Caption
\usepackage{titlesec} %Manipulation der Headings

% Grafiken im Header
%-----------------------------------------------------------------
\usepackage{fancyhdr} %Header Seiten, Titel etc.
%\pagestyle{fancyplain}
%\lhead{\begin{picture}(0,0) \put(0,0){\includegraphics[width=3.5cm]{imbei}} \end{picture}}
%\rhead{\begin{picture}(0,0) \put(0,0){\includegraphics[width=2.5cm]{mse}} \end{picture}}
%\pagestyle{fancy}
%\fancyhf{}
%\fancyhead[L]{\includegraphics[height=25mm,width=35mm]{mse}}
%\fancyhead[R]{\includegraphics[height=25mm, width=60mm]{imbei}}
%\renewcommand{\headrulewidth}{0pt}
%\renewcommand{\footrulewidth}{0pt}
%\setlength\headheight{117.89105pt}

%\usepackage{helvet}

 
%\makeatother

% Theorems
%-----------------------------------------------------------------
\newtheorem{thm}{Theorem}[section]
\newtheorem{cor}[thm]{Corollary}
\newtheorem{lem}[thm]{Lemma}
\newtheorem{prop}[thm]{Proposition}
\theoremstyle{definition}
\newtheorem{defn}[thm]{Definition}
\theoremstyle{remark}
\newtheorem{rem}[thm]{Remark}

% Shortcuts.
% One can define new commands to shorten frequently used
% constructions. As an example, this defines the R and Z used
% for the real and integer numbers.
%-----------------------------------------------------------------
\def\RR{\mathbb{R}}
\def\ZZ{\mathbb{Z}}

% Similarly, one can define commands that take arguments. In this
% example we define a command for the absolute value.
% -----------------------------------------------------------------
%\newcommand{\abs}[1]{\left\vert#1\right\vert}
%\renewcommand*\familydefault{\sfdefault}
%\newcommand{\periodafter}[1]{#1.}
%\titlelabel{\thetitle.\quad}


% Operators
% New operators must defined as such to have them typeset
% correctly. As an example we define the Jacobian:
% -----------------------------------------------------------------
\DeclareMathOperator{\Jac}{Jac}

%-----------------------------------------------------------------
%\titlehead{
%\begin{flushleft}
 %\includegraphics[height=25mm,width=35mm]{mse}
%\hfill
%\includegraphics[height=25mm, width=60mm]{imbei}
%\end{flushleft}

%}
\title{Hausaufgabe Praxisbeispiele zur Biometrie}

\author{Marvin Heyne\\
  	\small marvinheyne@gmail.com \\
}
\begin{document}

\maketitle
\tableofcontents
\noindent


\section{Schwächen im Design}
\begin{enumerate}
	\item[a) ]Laut den Aussagen im Manuskriptentwurf wurde ein Test auf Varianzhomogenität vorgenommen: 
	\begin{quotation}
		If the pre-test for equality of variances was not significant, a Welch's ANOVA analysis was conducted.
	\end{quotation}
	Die beschriebene Art und Weise im Vorgehen ist leider nicht korrekt. Der (Levene-) Test auf Homogenität der Varianzen, eine zentrale Voraussetzung für die Anwendung einer ANOVA, prüft die $H_0: \sigma^{2}_{1} = \sigma^{2}_{2} = \dots = \sigma^{2}_k$. Sollte sich anhand des Test ein $p-Wert>0.05$ ergeben\footnote{Strenggenommen fehlt diese Information ebenfalls im Manuskript.}, so wird die \underline{$H_0$ nicht abgelehnt} und eine ANOVA unter der Annahme von homogenen Varianzen wäre anwendbar. Hinzu kommt, dass das Testen von Voraussetzungen prinzipiell unter diversen Schwächen leidet. Genannt seien hier die folgende Alpha-Inflation sowie das Umkehren der Logik des statistischen Testens (der Nachweis von Nicht-Vorhandensein eines Unterschieds). An dieser Stelle käme eine grafische Analyse der Varianzhomogenität mittels eines \textit{Residual vs. Fitted} Plot in Frage. Allgemein ist die einfaktorielle ANOVA  robust gegen \textit{leichte} Verletzungen ihrer Annahmen, jedoch könnte im vorliegenden Setting die Welch-ANOVA  dennoch zum Einsatz kommen, da die Gruppen in ihrer Größe stark unterschiedlich sind und eine Verletzung der Varianzhomogenität dann zu einer starken Verzerrung des F-Tests führt.
	\item[b) ]Mögliche weitere, die Wahrscheinlichkeit für das Auftreten einer \textit{nocturnal hypoxemia} beeinflussende Faktoren wie Alter, Geschlecht, Raucherstatus oder Komorbiditäten wie COPD und Typ-2-Diabetes finden augenscheinlich in der Datenerhebung keine Berücksichtigung. Dies könnte den beobachtenden Effekt zwischen den Gruppen verzerren, wenn bspw. in der C19-Gruppe verstärkt Patienten mit bereits bestehenden chronischen Atemwegserkrankungen zu finden sind.
\end{enumerate}
\section{Schwächen der Analysemethode}
\begin{enumerate}
	\item[a) ] Die durchgeführte Auswertung berücksichtigt nicht die Ausgangswerte der Gruppen, sowie die Abhängigkeitsstruktur der Messwerte (i) innerhalb eines Patienten aufgrund von Messwiederholungen oder (ii) innerhalb eines Pneumonietypus. Dies kann zu einer Erhöhung der Wahrscheinlichkeit falsch-positiver sowie falsch-negativer Ergebnisse führen. Die Auswertung sollte mittels eines  \textit{mixed model for repeated measures} durchgeführt werden. Diese Form des Mehrebenenmodells würde es ermöglichen die Abhängigkeitsstruktur der Cluster \textit{Patienten} und \textit{Pneumonietypus} zu berücksichtigen und korrekt zu modellieren.

\end{enumerate}


\section{Schwächen der Schlussfolgerung}




% Bibliography
%-----------------------------------------------------------------

%
%\begin{thebibliography}

%\bibliographystyle{plainnat}
%\bibliography{expose}

%\end{thebibliography}


\end{document}
