%%%%%%%%%%%%%%%%%%%%%%%%%%%%%%%%%%%%%%%%%%%%%%%%%%%%%%%%%%%%%%%%%%%%%%%%%%%
%
% Template for a LaTex article in English.
%
%%%%%%%%%%%%%%%%%%%%%%%%%%%%%%%%%%%%%%%%%%%%%%%%%%%%%%%%%%%%%%%%%%%%%%%%%%%

\documentclass{scrartcl}

% AMS packages:
\usepackage{amsmath, amsthm, amsfonts} %Mathe-Pakete
\usepackage[colorlinks=false, pdfborder={0 0 0}]{hyperref} %Klickbare Referenzierung
\usepackage[hang,flushmargin]{footmisc} %Manipulation der Footnotes
\usepackage[square,sort,comma,numbers]{natbib} %Literaturverzeichnis mit Vancouver
\usepackage[ngerman]{babel}
\usepackage[utf8]{inputenc} %Zeichenerkennung
\usepackage{geometry} %Seiten Layout Paket
\geometry{a4paper,left=40mm,right=40mm, top=1cm, bottom=2cm} %Manipulation Layout 

% Change font of title and section
\setkomafont{disposition}{\normalfont}
%\addtokomafont{disposition}{\rmfamily}
\usepackage{setspace}%zeilenabstand
\onehalfspacing %1,5-facher zeilenabstand
%\usepackage{lmodern}
\usepackage{microtype}%wichtig für Blockabsatz
\usepackage{alnumsec}
\usepackage{anysize}
\usepackage{parskip}%Einzug&Absatz
\setlength{\parindent}{1em}%mit Einzug
\usepackage{graphicx} %Grafiken
\usepackage{float} %Float-Umgebung
\usepackage{longtable} %Tabellen über zwei Seiten
\usepackage{caption} %Float ohne Caption
\usepackage{titlesec} %Manipulation der Headings
\usepackage{tocstyle}
\usetocstyle{allwithdot}

% Grafiken im Header
%-----------------------------------------------------------------
\usepackage{fancyhdr} %Header Seiten, Titel etc.
%\pagestyle{fancyplain}
%\lhead{\begin{picture}(0,0) \put(0,0){\includegraphics[width=3.5cm]{imbei}} \end{picture}}
%\rhead{\begin{picture}(0,0) \put(0,0){\includegraphics[width=2.5cm]{mse}} \end{picture}}
%\pagestyle{fancy}
%\fancyhf{}
%\fancyhead[L]{\includegraphics[height=25mm,width=35mm]{mse}}
%\fancyhead[R]{\includegraphics[height=25mm, width=60mm]{imbei}}
%\renewcommand{\headrulewidth}{0pt}
%\renewcommand{\footrulewidth}{0pt}
%\setlength\headheight{117.89105pt}

%\usepackage{helvet}

 
%\makeatother

% Theorems
%-----------------------------------------------------------------
\newtheorem{thm}{Theorem}[section]
\newtheorem{cor}[thm]{Corollary}
\newtheorem{lem}[thm]{Lemma}
\newtheorem{prop}[thm]{Proposition}
\theoremstyle{definition}
\newtheorem{defn}[thm]{Definition}
\theoremstyle{remark}
\newtheorem{rem}[thm]{Remark}

% Shortcuts.
% One can define new commands to shorten frequently used
% constructions. As an example, this defines the R and Z used
% for the real and integer numbers.
%-----------------------------------------------------------------
\def\RR{\mathbb{R}}
\def\ZZ{\mathbb{Z}}

% Similarly, one can define commands that take arguments. In this
% example we define a command for the absolute value.
% -----------------------------------------------------------------
%\newcommand{\abs}[1]{\left\vert#1\right\vert}
%\renewcommand*\familydefault{\sfdefault}
%\newcommand{\periodafter}[1]{#1.}
%\titlelabel{\thetitle.\quad}


% Operators
% New operators must defined as such to have them typeset
% correctly. As an example we define the Jacobian:
% -----------------------------------------------------------------
\DeclareMathOperator{\Jac}{Jac}

%-----------------------------------------------------------------
%\titlehead{
%\begin{flushleft}
 %\includegraphics[height=25mm,width=35mm]{mse}
%\hfill
%\includegraphics[height=25mm, width=60mm]{imbei}
%\end{flushleft}

%}
\title{Hausaufgabe Praxisbeispiele zur Biometrie}

\author{Marvin Heyne\\
  	\small marvinheyne@gmail.com \\
}

\begin{document}

\maketitle
\tableofcontents
\noindent

\section{Schwächen im Design}
\begin{enumerate}
	\item[a) ]Es findet keine Erhebung möglicher Confounder statt. Mögliche weitere, die Wahrscheinlichkeit für das Auftreten einer \textit{nocturnal hypoxemia} beeinflussende Faktoren wie Alter, Geschlecht, Raucherstatus oder Komorbiditäten wie COPD und Typ-2-Diabetes finden augenscheinlich in der Datenerhebung keine Berücksichtigung. Dies könnte den beobachtenden Effekt zwischen den Gruppen verzerren, wenn bspw. in der C19-Gruppe verstärkt Patienten mit bereits bestehenden chronischen Atemwegserkrankungen zu finden sind. Im Regelfall sollten die zu erhebenden Faktoren und Confounder bereits Teil der Dokumentation im Studienprotokoll sein. Da es sich hier um eine retrospektive Datenerhebung und Auswertung von Patientenakten handelt, sollten diese Daten definitiv noch erhoben werden. 
	\item[b) ]Zwar findet eine Erhebung der unterschiedlichen Beatmungsarten statt, jedoch fehlt letztendlich der zeitliche Rahmen bzw. die Dauer, unter der der jeweilige Patient eine bestimmte Art der Beatmung erhalten hat. Erfasst wird nur, ob der jeweilige Patient invasiv oder nicht-invasiv beatmet wurde. Die Dauer der Behandlung mit einer bestimmten Art der Beatmung orientiert sich an Bedürfnissen des jeweiligen Patienten und daher sollte dies in der Datenerhebung und Auswertung (Adjustierung) berücksichtigt werden.    
	\item[c) ]Die retrospektive Datenerhebung bzw. der Einschluss nur von Patienten, die letztlich invasiv beatmet werden mussten, birgt das Risiko für einen Selektionsbias. Patienten, deren Status sich im Rahmen der nicht-invasiven Beatmung soweit verbessern konnte, dass diese entlassen wurden, waren nicht Teil der Datenerhebung und somit auch nicht der Auswertung. Dies könnte den beobachteten Behandlungseffekt der invasiven Beatmung verzerren, da Patienten mit schwereren Pneumonie-Formen möglicherweise stärker von der Intervention profitieren. Entweder sollte deutlicher beschrieben werden, dass sich die Schlussfolgerungen auch nur auf das eingeschlossene Patientenkollektiv beziehen oder aber der Einschluss der Patienten in die Datenerhebung muss nach der Diagnose einer Pneumonie unabhängig vom Beatmungsstatus erfolgen. 
	\item[d) ]Der genaue Zeitpunkt der Erhebung des Zielparameters T90 wurde a priori nicht einheitlich definiert. Es ist daher unklar zu welchem Zeitpunkt beim Patienten innerhalb der jeweiligen Beatmungsphase der T90-Wert erhoben wurde. Dies kann zu Verzerrungen der beobachteten Unterschiede führen. Sollten die Werte jeweils am Ende der Beatmungsphase erhoben worden sein, dann vermutlich zu einem Zeitpunkt, als es dem Patienten bereits so schlecht ging, dass eine Wechsel von bspw. nicht-invasiver zu invasiver Beatmung unvermeidlich war. 
	\item[e) ] Die Einschlusskriterien der Studie sind anhand der Studiendaten nicht nachvollziehbar. Im Text finden sich nur die Informationen, dass Patienten mit unzureichender Sauerstoffsättigung zunächst nicht-invasiv und bei weiterer Verschlechterung des Allgemeinzustandes dann invasiv beatmet werden. Hierbei wird weder einheitlich definiert was unzureichende Sauerstoffsättigung ist, noch aufgeklärt, wie bewertet wird, wann der Patient hinsichtlich seiner Beatmung eskaliert werden sollte. Weiterhin sind auch keine sonstigen Ein- und Ausschlusskriterien für das Kollektiv bzw. die Datenerhebung beschrieben.   
\end{enumerate}
\section{Schwächen der Analysemethode}
\begin{enumerate}
	\item[a) ]Laut den Aussagen im Manuskriptentwurf wurde ein Test auf Varianzhomogenität vorgenommen: 
	\begin{quotation}
		If the pre-test for equality of variances was not significant, a Welch's ANOVA analysis was conducted.
	\end{quotation}
	Die beschriebene Art und Weise im Vorgehen ist leider nicht korrekt. Der (Levene-) Test auf Homogenität der Varianzen, eine zentrale Voraussetzung für die Anwendung einer ANOVA, prüft die $H_0: \sigma^{2}_{1} = \sigma^{2}_{2} = \dots = \sigma^{2}_k$. Sollte sich anhand des Test ein $p-Wert>0.05$ ergeben\footnote{Strenggenommen fehlt diese Information ebenfalls im Manuskript.}, so wird die \underline{$H_0$ nicht abgelehnt} und eine ANOVA unter der Annahme von homogenen Varianzen wäre anwendbar. Das Testen von Verteilungs-Voraussetzungen statistischer Verfahren leidet jedoch prinzipiell unter diversen Schwächen. Genannt seien hier die folgende Alpha-Inflation sowie das Umkehren der Logik des statistischen Testens (der Nachweis von Nicht-Vorhandensein eines Unterschieds). An dieser Stelle käme eine grafische Analyse der Varianzhomogenität mittels eines \textit{Residual vs. Fitted} Plot in Frage. Allgemein ist die einfaktorielle ANOVA  robust gegen \textit{leichte} Verletzungen ihrer Annahmen, jedoch könnte im vorliegenden Setting die Welch-ANOVA  dennoch zum Einsatz kommen, da die Gruppen in ihrer Größe stark unterschiedlich sind und eine Verletzung der Varianzhomogenität dann zu einer starken Verzerrung des F-Tests führt.
	\item[b) ] Die durchgeführte Auswertung berücksichtigt nicht die Ausgangswerte der Gruppen, sowie die Abhängigkeitsstruktur der Messwerte (i) innerhalb eines Patienten aufgrund von Messwiederholungen oder (ii) innerhalb eines Pneumonietypus. Dies kann zu einer Erhöhung der Wahrscheinlichkeit falsch-positiver sowie falsch-negativer Ergebnisse führen. Die Auswertung sollte mittels eines  \textit{mixed model for repeated measures} durchgeführt werden. Diese Form des Mehrebenenmodells würde es ermöglichen die Abhängigkeitsstruktur der Cluster \textit{Patienten} und \textit{Pneumonietypus} zu berücksichtigen und korrekt zu modellieren.
	\item[c) ]Insgesamt werden 18 statistische Tests durchgeführt, was aufgrund der fehlenden Adjustierung des $\alpha$-Niveaus zu einer Inflation des Typ-I-Fehlers führt. Es sollte zunächst überlegt werden, ob alle diese Fragestellungen zwingend mit Hypothesentests beantwortet werden sollen. Im Falle mehrerer inferenzstatistischer Tests sollte eine Testprozedur gewählt werden, die in der Lage ist, dass vorab festgelegte $\alpha$-Niveau einzuhalten. Hier käme bspw. ein hierarchisches Testprozedere, bei dem eine a priori festgelegte Reihenfolge von Testhypothesen nacheinander statistischen Tests unterworfen wird, in Frage. Weitere Methoden, die in der Lage wären das spezifizierte $\alpha$-Niveau einzuhalten sind die Testprozedur nach Wiens oder eine Adjustierung nach Bonferroni-Holm.
	\item[d) ]Die in der Tabelle berichteten Mittelwerte und Standardabweichungen lassen eine Normalverteilung zwar als nicht unwahrscheinlich erscheinen, jedoch ergeben Fragen zum erfassten Wertebereich sowie der Qualität der Datenerhebung. Folgend den ersten neun berichteten Mittelwerten und Standardabweichungen von T90-Werten, die als Anteil der nächtlichen Schlafzeit mit einer $O_2$-Sättigung $\le 90\%$ in Minuten definiert ist, flossen negative Werte in die Datenbank mit ein. Dies ist im Falle von Anteilswerten ein nicht zulässiger Wertebereich und daher sollte ein weiterer Abgleich der Datenbank mit den Quelldaten erfolgen.
	\item [e) ]Die p-Werte der letzten Spalte der Tabelle, die augenscheinlich aus einer ANOVA pro Messzeitpunkt (nA vs. NIV vs. IV) stammen, sind nicht Teil der im Text beschriebenen Fragestellung und sollten weder durchgeführt noch berichtet werden. Folgend der Fragestellung liegt das Interesse der Forschenden in der Verbesserung der nächtlichen T90-Werte, was primär über das $\delta$ der Veränderung beschrieben wird. Weiterhin wird das Prinzip der ANOVA dann noch dazu missbraucht, innerhalb eines Pneumonietypus die unterschiedlichen Messzeitpunkte als Gruppen gegeneinander zu testen. Dieses Vorgehen entspricht nicht einem logischen, methodischen Ansatz zur Berechnung von Unterschieden. Hier wäre das bereits beschriebene MMRM ein zu bevorzugender analytischer Ansatz, der im Vergleich zu dem von den Autoren gewählten auch eine höhere Power hätte.   
\end{enumerate}


\section{Schwächen der Schlussfolgerung}

\begin{enumerate}
	\item[a) ]Eine der Schlussfolgerungen der Autoren lautet:
		\begin{quotation}
		Our data suggest that other way of oxygen supply/ventilation are required for pneumonia patients with virus infection [$\ldots$].
	\end{quotation}
			  Diese Aussage stimmt nicht mit den berichteten Daten überein. In allen drei Pneumonie-Gruppen reduziert sich der T90-Wert unter nicht-invasiver bzw. invasiver Beatmung im Vergleich zum Werte nach der Aufnahme. Zwar ergibt sein kein statistisch-signifikantes Ergebnis, jedoch ist die Ableitung, dass eine \textit{nocturnal hypoxemia} nicht mit invasiver bzw. nicht-invasiver Beatmung reduziert werden kann nicht korrekt. Ein nicht-statistisches Ergebnis ist keine Evidenz für das Nicht-Vorhandensein eines Effektes. Hierbei sei insbesondere angemerkt, dass die Fallzahl in der Gruppe C19 mit $n=19$ im Vergleich zu den beiden anderen Gruppen extrem klein ist und somit schlichtweg zu gering war, um einen statistisch-signifikanten Effekt nachzuweisen.
	 \item[b) ]Reüssierend auf die bereits angesprochene Schwäche hinsichtlich der Einschlusskriterien, wird eine zu verallgemeinernde Aussage über den Effekt gemacht. Die Tatsache, dass nur Patienten eingeschlossen und ausgewertet wurden die letztlich invasiv beatmet wurden, muss in der Schlussfolgerung bzw. allgemeinen Diskussion der Ergebnisse als Limitation berücksichtigt werden. 
	 \item[c) ]Zur Schlussfolgerungen sollte mindestens auch die Veränderung pro Pneumonietypus Berücksichtigung finden. Dies entspricht der primären Fragestellung, die im einleitenden Text beschrieben wird. 
\end{enumerate}


% Bibliography
%-----------------------------------------------------------------

%
%\begin{thebibliography}

%\bibliographystyle{plainnat}
%\bibliography{expose}

%\end{thebibliography}


\end{document}
